\documentclass{article}
\usepackage[T1]{fontenc}
\usepackage[polish]{babel}
\usepackage[utf8]{inputenc}
\usepackage{amssymb}
\usepackage[a4paper,top=2cm,bottom=2cm,left=3cm,right=3cm,marginparwidth=1.75cm]{geometry}
\usepackage{amsmath}
\usepackage{graphicx}

\title{Zadanie egzaminacyjne z RPiS 3}
\author{Mikołaj Drozd 339139}
\date{09.05.2024}

\begin{document}

\maketitle
\section{Cel zadania}
\large
Mamy dany rozkład wykładniczy Exp($\lambda$), którego gęstość określona jest wzorem:\\
\begin{center}
   f(x)=$\lambda e^{-\lambda x}$ 
\end{center}
Musimy wyznaczyć MGF tego rozkładu, następnie oszacować $P(X \geqslant\lambda a)$ za pomocą nierówności Markova, Chebysheva i Chernoffa przy założeniu, że X $\sim$ Exp($\lambda$). Ostatnim krokiem będzie sporządzenie tabeli zawierającej wartości dokładne oraz oszacowania dla k=9, m=3, $\lambda$=13 oraz a $\in \{3,4,6,10\}$.

\section{Wyznaczenie MGF}
\Large
$M_X(t) = E(e^{t X}) = \int_0^\infty e^{t x}  {f_X}(x)dx =\lambda \int_0^\infty e^{x(t-\lambda)}dx = \lambda \int_0^\infty e^s \frac 1 {t-\lambda}ds = \frac \lambda {t-\lambda} \int_0^\infty e^s ds = \frac \lambda {t-\lambda} e^{x (t-\lambda)}|_0^\infty$ \\ \\
Rozpatrzmy przypadki: \\
a) $t<\lambda$, wtedy całka jest zbieżna \\
b) $t=\lambda$, wtedy całka jest rozbieżna, ponieważ w mianowniku mamy 0 \\
c) $t>\lambda$, wtedy całka jest również rozbieżna, ponieważ dążymy do nieskończoności \\ \\
Zatem dla $t<\lambda$ mamy $M_X(t) = \frac \lambda {t-\lambda} [0 - 1] = - \frac \lambda {t-\lambda} = \frac \lambda {\lambda - t}$

\section{Oszacowania Markova, Chebysheva i Chernoffa}
\Large
\subsection{Markov}
$P(X \geqslant\lambda a ) \leqslant \frac {E(X)} {\lambda a} = \frac {\frac 1 \lambda} {\lambda a} = \frac 1 {\lambda^2 a}$ 
\subsection{Chebyshev}
$P(|X - \frac 1 \lambda| \geqslant b) \leqslant \frac {V(X)} {b^2}$ \\
$P(|X|) \geqslant b + \frac 1 \lambda) \leqslant \frac {V(X)} {b^2}$ \\
$\lambda a = b + \frac 1 \lambda \\
b = \lambda a - \frac 1 \lambda $\\
Zatem: \\
$P(X \geqslant\lambda a) \leqslant \frac {V(X)}  {(\lambda a - {\frac 1 \lambda})^2} = \frac {\frac 1 {\lambda^2}} {(\lambda a - {\frac 1 \lambda})^2} = \frac {\frac 1 {\lambda^2}} {\lambda^2 a^2 - 2 a + \frac 1 {\lambda^2}} = \frac 1 {\lambda^4 a^2 - 2 a {\lambda^2} + 1} $
\subsection{Chernoff}
$P(X \geqslant\lambda a) \leqslant e^{-\lambda a t} M_X(t) = e^{-\lambda a t} \frac \lambda {\lambda - t} = f(t)$

\section{Wartości dokładne i oszacowania}
\Large
Wartość dokładna: \\
$P(X \geqslant\lambda a) = 1 - P(X < \lambda a) = 1 - (1 - e^{-{\lambda^2} a}) = e^{-{\lambda^2} a} = e^{-169 a}$ \\ \\
Wartość oszacowania Markova: \\
$P(X \geqslant\lambda a ) \leqslant \frac 1 {\lambda^2 a} = \frac 1 {169 a}$ \\ \\
Wartość oszacowania Chebysheva: \\
$P(X \geqslant\lambda a ) \leqslant \frac 1 {28561 a^2 - 169 a + 1}$ \\ \\
Wartość oszacowania Chernoffa (szukamy minimum wyznaczonej wyżej funkcji): \\
$(e^{- \lambda a t} \frac \lambda {\lambda -t})' = 0 \\
e^{- \lambda a t} (- \lambda a) \frac \lambda {\lambda - t} + e^{- \lambda a t} \frac \lambda {({\lambda - t})^2} = 0 \\
- \lambda a + \frac 1 {\lambda - t} = 0$ \\
Stąd $t = \lambda - \frac 1 {\lambda a} = 4 - \frac 1 {4 a}$ \\
Czyli $P(X \geqslant\lambda a ) \leqslant e^{- \lambda a ({13 - \frac 1 {13 a}})} \frac \lambda {\lambda - (13 - \frac 1 {13 a})} = \frac {13} {\frac 1 {13 a} e^{13 a (13 - {\frac 1 {13 a}})}}$ \\ \\
Tabela (wartości w przybliżeniu): \\
\begin{center}
\begin{tabular}{ |c|c|c|c|c|}
 \hline 
 wartość\textbackslash a & 3 & 4 & 6 & 10 \\
 \hline\hline
 dokładna & $e^{-507}$ & $e^{-676}$ & $e^{-1014}$ & $e^{-1690}$\\
 \hline
 Markov & $ \frac{1}{507}$ & $ \frac{1}{676}$ & $ \frac{1}{1014}$ & $ \frac{1}{1690}$\\
 \hline
 Chebyshev & $ \frac{1}{256543}$ & $ \frac{1}{456301}$ & $ \frac{1}{1027183}$ & $ \frac{1}{2854411}$\\
 \hline
 Chernoff & $ \frac {507}{e^{506}}$ & $ \frac{676}{e^{675}}$ & $ \frac {1014} {e^{1013}}$ & $ \frac{1690}{e^{1689}}$\\
 \hline
\end{tabular}
\end{center}
\end{document}
